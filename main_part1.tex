\documentclass[12pt]{article}
\usepackage[utf8]{inputenc}
\usepackage[T1]{fontenc}
\usepackage[english]{babel}
\usepackage{amsmath, amssymb, amsthm}
\usepackage{geometry}
\usepackage{titling}
\usepackage{fancyhdr}
\usepackage{lipsum}
\usepackage{parskip}
\usepackage{forest}
\usepackage{tikz}
\usepackage{stmaryrd}
\usepackage{listings}
\usepackage{graphicx}
\usepackage{float}
\usepackage{alphalph}
\usepackage{cancel}
\usepackage{textgreek}
\usepackage{titlesec}
\usepackage{dsfont}
\usepackage{caption}
\usepackage{cancel}

\geometry{top=4cm, bottom=4cm, left=4cm, right=4cm}
\pagestyle{fancy}
\fancyhf{}
\rhead{Pierre Pili $\cdot$ Marie Gardie $\cdot$ Isée Biglietti}
\lhead{Econometrics 3}
\cfoot{\thepage}
\setlength{\headheight}{14.49998pt}
\addtolength{\topmargin}{-2.49998pt}

\titleformat{\section}{\normalsize\bfseries}{\thesection}{1em}{}
\titleformat{\subsection}{\small\bfseries}{\thesubsection}{1em}{}

\renewcommand{\thesubsection}{(\alph{subsection})}
\renewcommand{\thesection}{\arabic{section}}

\title{Econometrics 3 - Problem Set 10}
\author{PILI Pierre $\cdot$ GARDIE Marie $\cdot$ BIGLIETTI Isée}
\date{\today}



\begin{document}
\maketitle

\section{Univariate analysis}
\subsection{Data presentation: source and plots for each time series}
U.S. Infra-Annual Labor Statistics: Unemployment Rate Total: From 15 to 64 Years for United States
Units: Percent, Seasonally Adjusted 
Frequency: Quarterly
From: 1970-01-01 // 2024-01-01
Source: Organization for Economic Co-operation and Development (le lien à ajouter https://www.oecd.org)
U.S. Gross Domestic Product
Description: Gross domestic product (GDP), the featured measure of U.S. output, is the market value of the goods and services 
produced by labor and property located in the United States.
Units: Billions of Dollars, Seasonally Adjusted Annual Rate
Frequency: Quarterly
From: 1970-01-01 // 2024-01-01
Source: U.S. Bureau of Economic Analysis (le lien à ajouter https://www.bea.gov)

\subsection{Unit root and stationarity tests for each time series}

\subsection{Identification of the ARMA or ARIMA process for two time series}

\subsection{Forecasts: in-sample and out-of-sample, for two time series}

\end{document}