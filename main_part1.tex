\documentclass[12pt]{article}
\usepackage[utf8]{inputenc}
\usepackage[T1]{fontenc}
\usepackage[english]{babel}
\usepackage{amsmath, amssymb, amsthm}
\usepackage{geometry}
\usepackage{titling}
\usepackage{fancyhdr}
\usepackage{lipsum}
\usepackage{parskip}
\usepackage{forest}
\usepackage{tikz}
\usepackage{stmaryrd}
\usepackage{listings}
\usepackage{graphicx}
\usepackage{float}
\usepackage{alphalph}
\usepackage{cancel}
\usepackage{textgreek}
\usepackage{titlesec}
\usepackage{dsfont}
\usepackage{caption}
\usepackage{cancel}
\usepackage{hyperref}

\geometry{top=4cm, bottom=4cm, left=4cm, right=4cm}
\pagestyle{fancy}
\fancyhf{}
\rhead{Pierre Pili $\cdot$ Marie Gardie $\cdot$ Isée Biglietti}
\lhead{Econometrics 3}
\cfoot{\thepage}
\setlength{\headheight}{14.49998pt}
\addtolength{\topmargin}{-2.49998pt}

\titleformat{\section}{\normalsize\bfseries}{\thesection}{1em}{}
\titleformat{\subsection}{\small\bfseries}{\thesubsection}{1em}{}

\renewcommand{\thesubsection}{(\alph{subsection})}
\renewcommand{\thesection}{\arabic{section}}

\title{Econometrics 3 - Problem Set 10}
\author{PILI Pierre $\cdot$ GARDIE Marie $\cdot$ BIGLIETTI Isée}
\date{\today}



\begin{document}
\maketitle

\section{Univariate analysis}
\subsection{Data presentation: source and plots for each time series}
Our two datasets can be presented by the following information : 
\begin{itemize}
    \item U.S. Infra-Annual Labor Statistics
    \begin{itemize}
        \item Description: U.S. total unemployment rate, from 15 to 64 years
        \item Units: Percent, seasonally adjusted
        \item Frequency: Quarterly
        \item From: 1970-01-01 // 2024-01-01
        \item Source: \href{https://www.oecd.org}{Organization for Economic Co-operation and Development}
    \end{itemize}
    \item U.S. Gross Domestic Product
    \begin{itemize}
        \item Description: Gross domestic product (GDP), the featured measure of U.S. output, is the market value of the goods and services 
        produced by labor and property located in the United States
        \item Units: Billions of Dollars, seasonally adjusted annual rate
        \item Frequency: Quarterly
        \item From: 1970-01-01 // 2024-01-01
        \item Source: \href{https://www.bea.gov}{U.S. Bureau of Economic Analysis}
    \end{itemize}
\end{itemize}

When plotting our two raw time series, we can say that GDP is clearly upward trending, with a time-varying mean, so not stationary at all 
(see Figure \ref*{fig:GDP_1}), while unemployment rate has no trend but is persistent, clearly time dependent, but more likely to be stationary
(see Figure \ref*{fig:urate_1}).\\

\begin{figure}[H]
    \centering
    \includegraphics[width=0.5\textwidth]{OUTPUT/GDP_1.png}
    \caption{GDP evolution: raw.}
    \label{fig:GDP_1}
\end{figure}

\begin{figure}[H]
    \centering
    \includegraphics[width=0.5\textwidth]{OUTPUT/urate_1.png}
    \caption{Unemployment rate evolution: raw.}
    \label{fig:urate_1}
\end{figure}

We take the log of the GDP because of the shape of GDP, to expose its trend more linearly,
and we take its first and second differenciation, the first and second differenciation of raw GDP, as well as the first and second 
differenciation for the unemployment rate. \\
Looking at the plots (see Figures \ref*{fig:lGDP}, \ref*{fig:dlGDP}, \ref*{fig:ddlGDP}, \ref*{fig:dGDP}, \ref*{fig:ddGDP}, \ref*{fig:drate} 
and \ref*{fig:ddrate}), we will continue on using the raw unemployment rate, the logarithm of GDP and the first difference of the 
logarithm of GDP to test unit root and stationarity, as they are the more probable to be stationary. 

\begin{figure}[H]
    \centering
    \includegraphics[width=0.5\textwidth]{OUTPUT/lGDP.png}
    \caption{Log GDP evolution}
    \label{fig:lGDP}
\end{figure}
\begin{figure}[H]
    \centering
    \includegraphics[width=0.5\textwidth]{OUTPUT/dlGDP.png}
    \caption{Economic Growth}
    \label{fig:dlGDP}
\end{figure}
\begin{figure}[H]
    \centering
    \includegraphics[width=0.5\textwidth]{OUTPUT/ddlGDP.png}
    \caption{Fist differenciation of economic growth}
    \label{fig:ddlGDP}
\end{figure}
\begin{figure}[H]
    \centering
    \includegraphics[width=0.5\textwidth]{OUTPUT/dGDP.png}
    \caption{Fist differenciation of GDP evolution}
    \label{fig:dGDP}
\end{figure}
\begin{figure}[H]
    \centering
    \includegraphics[width=0.5\textwidth]{OUTPUT/ddGDP.png}
    \caption{Twicely differetiated GDP}
    \label{fig:ddGDP}
\end{figure}
\begin{figure}[H]
    \centering
    \includegraphics[width=0.5\textwidth]{OUTPUT/drate.png}
    \caption{Fist differenciation of unemployment rate}
    \label{fig:drate}
\end{figure}
\begin{figure}[H]
    \centering
    \includegraphics[width=0.5\textwidth]{OUTPUT/ddrate.png}
    \caption{Twicely differetiated unemployment rate}
    \label{fig:ddrate}
\end{figure}

\subsection{Unit root and stationarity tests for each time series}

\subsection{Identification of the ARMA or ARIMA process for two time series}

\subsection{Forecasts: in-sample and out-of-sample, for two time series}

\end{document}